\documentclass[margin,line]{res}

\usepackage[utf8]{inputenc}
\usepackage[normalem]{ulem}
\usepackage{todonotes}


\oddsidemargin -.5in
\evensidemargin -.5in
\textwidth=6.0in
\itemsep=0in
\parsep=0in
% if using pdflatex:
%\setlength{\pdfpagewidth}{\paperwidth}
%\setlength{\pdfpageheight}{\paperheight} 
\widowpenalties 1 10000

\newenvironment{list1}{
  \begin{list}{\ding{113}}{%
      \setlength{\itemsep}{0in}
      \setlength{\parsep}{0in} \setlength{\parskip}{0in}
      \setlength{\topsep}{0in} \setlength{\partopsep}{0in} 
      \setlength{\leftmargin}{0.17in}}}{\end{list}}
\newenvironment{list2}{
  \begin{list}{$\bullet$}{%
      \setlength{\itemsep}{0in}
      \setlength{\parsep}{0in} \setlength{\parskip}{0in}
      \setlength{\topsep}{0in} \setlength{\partopsep}{0in} 
      \setlength{\leftmargin}{0.2in}}}{\end{list}}



\begin{document}

\name{Luis Manuel Diaz Angulo, PhD \vspace*{.1in}}

\begin{resume}
\section{\sc Current Position}
 Assistant Professor
\section{\sc Contact Information}
\vspace{.05in}
\begin{tabular}{@{}p{2in}p{4in}}
Department of Electromagnetism   & {\it Work:}  (+34) 667-87-69-42 \\            
School of Sciences             & \\         
University of Granada            & \\
Granada, 18003, Spain            & {\it E-mail:}  lm@diazangulo.com\\       
\end{tabular}

\section{\sc Research Interests}
Computational Electromagnetics,
Electromagnetic Compatibility, 
Numerical methods, 
Galerkin methods,
Plasma technology,
High Performance Computing.

\section{\sc Skills} 
 {\bf Languages}
  \begin{list2}
   \item Spanish, mother tongue.
   \item English, fluent, B2 certification.
  \end{list2}
 {\bf Computer Related}
 \begin{list2} 
  \item Programming: C/C++, Python,  MatLab/Octave.
  \item Simulation/CAD: Ansys HFSS, FEKO, OpenFoam (Advanced certification), Semcad, CST Microwave Studio, GiD.
  \item Applications: Eclipse CDT, MS Visual Studio, Microsoft's Office, \LaTeX.
  \item Operating Systems: Linux, Windows.
  \end{list2}

\section{\sc Education}
{\bf University of Granada (UGR)}, Granada, Spain\\
\vspace*{-.1in}
\begin{itemize}
\item Doctor in Philosophy, Nov. 2014
 \begin{list2}
  \vspace*{.05in}
  \item[-] Dissertation Topic: \\{\it Discontinuous Galerkin Methods applied to Computational Electromagnetics}
  \item[-] Advisors:  Salvador G. Garcia, Mario F. Pantoja, Jesus G. Alvarez.
  \item[-] Grade of \textit{cum laude}
  \item[-] International mention.
 \end{list2}
\vspace*{.1in}
\item Major degree in Physics, Oct. 2003 - July 2008
 \begin{list2}
  \item[-] {\it Licenciado en Física.}
  \item[-] 2007/08 academic year spent in The University of Manchester, UK.
  \item[-] With special focus on electromagnetics and plasma technology.
 \end{list2}
 \vspace*{.1in}
\item Major degree in Electronics Engineering, February 2015
 \begin{list2}
  \item[-] {\it Licenciado en Ingeniería Electrónica.}
  \item[-]  Final project: {\it Simulation of a Dual Mode Circular Waveguide Filter with DGTD}.
 \end{list2}
\vspace*{.1in}
 \item Master in Physics and Mathematics, Oct. 2009 - Dec. 2010
 \begin{list2}
  \item[-] {\it Master en Física y Matemáticas (FISYMAT)}. 
  \item[-] Obtained the highest mark in the final research project.
 \end{list2}
\vspace*{.1in}
 \item Master in Secondary Education, Oct. 2014 - Sep. 2015
 \begin{list2}
  \item[-] {\it Máster Universitario en Profesorado de Enseñanza Secundaria Obligatoria y Bachillerato, Formación Profesional y Enseñanza de Idiomas}. 
  \item[-] Final project title: \textit{Calendars}.
 \end{list2}
\end{itemize}

\begin{minipage}{\textwidth}
{\bf Polytechnic University of Catalonia (UPC)}, Barcelona, Spain\\
\vspace*{-.1in}
\begin{itemize}
\item Summer School on Discontinuous Galerkin Methods,  Jun. 2012
\item Short Course on Terahertz technology and applications, May 2009
\end{itemize}
\end{minipage}

\begin{minipage}{\textwidth}
{\bf OpenFOAM}\\
\vspace*{-.1in}
\begin{itemize}
\item Foundation course and certification. Mar., 2014.
\item Advanced course and certification. May, 2015.
\end{itemize}
\end{minipage}


\section{\sc Academic Experience}
{\bf Universidad de Granada}, Granada, España\\
{\em PhD Advisor} \hfill {\bf Jan. 2016 - Sep. 2017}\\\\
\vspace*{-.25in}  
\begin{list2}
	\item I have co-advised a PhD Thesis titled: Development of FDTD algorithms for the solution of Maxwell Equations in Conformal Countours.
\end{list2}

{\bf Universidad de Granada}, Granada, España\\
{\em Acting substitute teacher, Assistant professor} \hfill {\bf Nov. 2015 - Present}\\
\vspace*{.05in}  
\begin{list2}
 \item \textit{Deterministic Computational Methods for Non-linear Physics} course for the Master of Physics and Mathematics (Fisymat)
 \item First year \textit{Physics II} course for the Master degree in Chemistry.
 \item First year \textit{Physics II} course for the Master degree in Chemical Engineering.
 \item First year \textit{Electromagnetism} course for the Master degree in Electronics Engineering.
 \item Third year \textit{Electromagnetism} course for the Master degree in Physics. 
\end{list2}

{\bf University of Granada}, Granada, Spain\\
{\em Instructor} \hfill {\bf Mar. 2012 - Sep. 2014}\\
Co-taught undergraduate level courses.  Shared responsibility for lectures, lab sessions, exams, coursework assignments, and grades.  
\vspace*{.05in}  
\begin{list2}
 \item First year \textit{Electromagnetism} course for the Master degree in Electronics Engineering.
 \item Third year \textit{Electromagnetism} course for the Master degree in Physics. 
\end{list2}

{\bf The Ohio State University - ElectroScience Laboratory}, Columbus, Ohio USA\\
\vspace{-.3cm}
{\em Research Internship} \hfill {\bf Jan. 2013 - Apr. 2013}\\
\begin{itemize}
 \item Three months in the The Ohio State University under the supervision of Professor Fernando
Teixeira.
 \item During this period I developed a new technique to improve the performance in full-wave
simulations that has been published in the {\it Journal of Computational Physics}.
\end{itemize}

{\bf Inst. National de Recherche en Informatique et en Automatique (INRIA)}, Nice, France\\
\vspace{-.3cm}
{\em Research Internship} \hfill {\bf Sep. 2012 - Dec. 2012}\\
\begin{itemize}
 \item Three months spent researching Space-Time Discontinuous Galerkin (STDG) Methods under the supervision of Stephane Lanteri.
 \item I lectured two sessions on Advanced Material Modelling for Electromagnetic Phenomena and STDG.
\end{itemize}

{\bf Int. Center for Numerical Methods in Engineering (CIMNE)}, Barcelona, Spain\\
\vspace{-.3cm}
{\em Research Internship} \hfill {\bf May 2010}\\
\begin{itemize}
 \item Two weeks developing a user interface for the UGRFDTD (www.sembahome.es)
\end{itemize}

{\bf Dassault Systems}, Saint Cloud, Paris, France\\
\vspace{-.3cm}
{\em Workshop instructor} \hfill {\bf Mar. 2012}\\
\begin{itemize}
 \item As part of the HIRF-SE project, I trained other project partners on how to use the FDTD
software developed by the University of Granada (UGRFDTD).
\end{itemize}


\section{\sc Scientific Reviewing}
I regularly review articles for the following scientific journals:\\
\begin{itemize}
 \item Journal Of Computational Physics
 \item IEEE Transactions On Microwave Theory and Techniques
 \item IEEE Antennas and Wireless Propagation Letters
 \item IEEE Transactions on Electromagnetic Compatibility
 \item ACES The Applied Computational Electromagnetics Society
 \item Progress In Electromagnetic Research - Journal of Electromagnetic Waves and Applications (PIER-JEMWA)
 \item Near Surface Geophysics
\end{itemize}

I have reviewed research projects for:
\begin{itemize}
	\item The Czech Science Foundation. 
	\item European Erasmus+ Program.
\end{itemize}


\section{\sc Research projects}
{\bf As coordinator}\\[0.15cm]
\begin{minipage}{\textwidth}
	{\em NiTest - Non Intrusive Testing} \hfill {\bf Sep. 2018 - Sep. 2020}\\
	This project aims to improve the industrial processes in Airbus. For this we evaluate and develop new non-intrusive testing methods for the different procedures that warrant the robustness and resilience of an aircraft against lightning impacts. This project is fully funded by Airbus.
\end{minipage}	

\begin{minipage}{\textwidth}
	{\em Development of low cost didactic experiments for the teaching of electromagnetics} \hfill {\bf Feb. 2017 - Feb. 2019}\\
	This project aims to improve the teaching-learning process during lectures and problem sessions, mainly for the physics courses in the Chemistry degree. To accomplish this, we developed new experimental demonstrations in the classroom to reinforce the theoretical-practical contents. A common characteristic of all experiments is its low cost, using commonly found elements.
\end{minipage}

\begin{minipage}{\textwidth}
  {\em Support and maintenance of SEMBA for NUDT} \hfill {\bf Jan. 2017 - Dec. 2018}\\
  This contract was signed between UGR and the National University of Defence and Technology (NUDT) from People's Republic of China. It includes the sale of a software license for the software SEMBA-UGRFDTD, technical support, training, and updates. 
\end{minipage}

\begin{minipage}{\textwidth}
  {\em Integration of the mesher ZMesher within GiD} \hfill {\bf Jan. 2017 - Dec. 2017}\\[0.1cm]
  This contract has regulated the sale of the exploitation rights, non-exclusively, of the ZMesher program, developed by the GEG-UGR to the \textit{Centre Internacional de Mètodes Numèrics a l'Enginyeria} (CIMNE). The contract includes support for its integration within the program GiD. 
\end{minipage}

{\bf As participant}\\[0.15cm]
\begin{minipage}{\textwidth}
	{\em Alhambra LFT} \hfill {\bf Sep. 2018 - Sep. 2021}\\
	En este proyecto se busca mejorar la eficiencia del simulador numérico SEMBA-UGRFDTD a la hora de tratar detalles finos del a estructura de un avión mediante el empleo de técnicas de submallado. El proyecto está financiado integramente por la empresa Airbus.
\end{minipage}	

{\em COST ACCREDIT} \hfill {\bf Jan. 2017 - Dec. 2019}\\
This project focuses on the challenges created by the stochastic and broadband nature of electromagnetic radiation in present and future multi-functional systems. It covers the modeling of stochastic field distributions and the experimental methods using near field probes. The project is a partnership between more than 20 participants of different countries in the European Union.

%\begin{minipage}{\textwidth}
%	{\em PRACE SREDIT} \hfill {\bf Enero 2017 - Diciembre 2019}\\
%	El proyecto SREDIT (Simulations of Radiated Emissions in Densely Integrated Technologies) incluye la participación de la UGR y tres entidades europeas más. Este proyecto se enmarca dentro de la iniciativa PRACE (Partnership for Advanced Computing in Europe) que busca mejorar el impacto científico de la investigación ofreciendo recursos computacionales a los participantes.
%\end{minipage}

\begin{minipage}{\textwidth}
  {\em UAVE-3} \hfill {\bf Jan. 2017 - Dec. 2019}\\
  The partners of this project (INTA, UGR, UPC) and participating companies (AIRBUS DS, ONERA y CIMNE) aim to develop innovative methodologies for the analysis and testing of Unmanned Air Vehicles (UAVs). This will be carried out by advanced methods for modeling and simulation which can be applied along the full life-cycle of the new generation of UAVs, including design, prototyping, certification, maintenance, and revision.
\end{minipage}

\begin{minipage}{\textwidth}
  {\em UAVEMI} \hfill {\bf Jan. 2014 - Dec. 2016}\\
  This project analyses the electromagnetic immunity of Unmanned Air Vehicles (UAV) under the effect of lightning and High Intensity Radiated Fields (HIRF). It is being developed by a parthernship between the Polytechnical University of Catalonia (UPC), the National Institute for Aerospace Technique (INTA) and the University of Granada.
\end{minipage}

\begin{minipage}{\textwidth}
  {\em MORFEO} \hfill {\bf Sep. 2014 - Dec. 2015}\\
  The MORFEO project aimed the assessment of the viability of new braiding techniques for the EMC protection of wires in aircrafts. It was co-funded by Airbus and the Andalusian regional government.
\end{minipage}

\begin{minipage}{\textwidth}
  {\em MANIAS} \hfill {\bf Dec. 2014 - Dec. 2015}\\
  The MANIAS project aims the development, fabrication and evaluation of multifunctional antennas integrated into aerodynamic surfaces. It was developed in partnership with the Airbus Group.
\end{minipage}

\begin{minipage}{\textwidth}
  {\em A-UGRFDTD} \hfill {\bf Aug. 2012 - Present}\\
  This project consists on the development of a fully functional geometrically conforming FDTD simulator.
  It is being funded by the Airbus Group and the University of Granada.
\end{minipage}

{\em Teralab} \hfill {\bf Nov. 2009 - Nov. 2013}\\
The TeraLab project aims to develop a numerical laboratory for the design and
simulation of Terahertz technology. To do so, physical models of materials in that
range of frequency are being developed and integrated in the numerical solvers.

{\em HIRF: SE} \hfill {\bf Dec. 2008 - Jun. 2013}\\
The High Intensity Radiated Fields: Synthetic Environment aims to develop a computer framework to simulate electromagnetic phenomena during the development phase. This project is funded by the European Comission and 44 other partners from the EU. The contribution by our group consisted in a fully functional FDTD volumic solver to be integrated within the framework. More information in: http://www.hirf-se.eu

{\em Terasense} \hfill {\bf Oct. 2008 - Mar. 2012}\\
The aim of Terasense (Terahertz Technology for Electromagnetic Sensing
applications) was to develop devices that, working in the range of frequency of
Terarhertzs, are able to detect objects or substances in real time. This project is
funded by the Spanish CONSOLIDER research program.

{\em Biological Effects of Radiofrequency} \hfill {\bf July 2008 - Dec. 2008}\\
The aim of this project was to assess the SAR of different biological tissues. During
the course of this project several CEM commercial tools were used, particularly
HFSS and Semcad. The project was funded by the European Defence Agency (EDA).

\begin{minipage}{\textwidth}
  {\em GPR techniques for the assessment of architectural heritage} \hfill {\bf July 2008 - Mar. 2010}\\
  In this project Ground Penetrating Radar (GPR) techniques were developed and employed to evaluate the state of conservation of some of the Andalusian architectural heritage. The project was funded by the Tourism and Commerce National Department and was carried on by the Marble Technological Center of Murcia and the University of Granada.
\end{minipage}

{\em Optimization techniques for UWB antennas} \hfill {\bf Oct. 2007 - Dec. 2010}\\
This project used CST software together with MatLab and in-house software to design UWB antennas. It was carried out by a partnership of four Spanish universities (UGR, UPC, UPV and UPCT) and funded by a national Research and Development program.

\end{resume}

\section{\sc Publications}
\begin{minipage}{\textwidth}
	Modeling and Measuring the Shielding Effectiveness of Carbon Fiber Composites.\\
	\textbf{Angulo, L. D.}; P. Gómez; B. Plaza; D. Poyatos; Cabello, M. R.; Bocanegra, D. E.; Garcia, S. G.\\
	{\it IEEE Journal on Multiscale and Multiphysics Computational Techniques}, \textbf{2019}, 67, 207-213.
\end{minipage}

\begin{minipage}{\textwidth}
	Application of Stochastic FDTD to Holland's Thin-Wire Method.\\
	Garcia, S. G.; Cabello, M. R.; \textbf{Angulo, L. D.}; Bretones, A. R.; M. G. Atienza; Pascual-Gil, E.; \\
	{\it IEEE Antennas and Wireless Propagation Letters}, \textbf{2019}, 18, 2046-2050.
\end{minipage}

\begin{minipage}{\textwidth}
	A New Conformal FDTD for lossy thin panels.\\
	Cabello, M. R.; \textbf{Angulo, L. D.}; Alvarez, J.;  Bretones, A. R.;  Garcia, S. G.\\
	{\it IEEE Transactions on Antennas and Propagation}, \textbf{2019}, 67, 7433-7439.
\end{minipage}
\begin{minipage}{\textwidth}
	SIVA UAV: A Case Study for the EMC Analysis of Composite Air Vehicles.\\
	Cabello, M. R.; Fernández, S.; Pous, M.; Pascual-Gil, E.; \textbf{Angulo, L. D.}; (17 autores más)\\
	{\it IEEE Transactions on Electromagnetic Compatibility}, \textbf{2017}, 59, 1103-1113.
\end{minipage}

\begin{minipage}{\textwidth}
	Multiresolution Time-Domain Analysis of Multiconductor Transmission Lines Terminated in Linear Loads.\\
	Tong, Z.; Sun, L.; Li, Y.; \textbf{Angulo, L. D.}; Garcia, S. G.; Luo, J.\\
	{\it Mathematical Problems in Engineering}, {\bf 2017}, 15 páginas.
\end{minipage}

\begin{minipage}{\textwidth}
	A hybrid Crank-Nicolson FDTD subgridding boundary condition for lossy thin-layer modeling IEEE Transactions on Microwave Theory and Techniques.\\
	Cabello, M. R.; \textbf{Angulo, L. D.}; Alvarez, J.; Flintoft, I.; (4 autores más)\\
	{\it IEEE Transactions on Microwave Theory and Techniques}, {\bf 2017}, 65, 1397-1406.
\end{minipage}

\begin{minipage}{\textwidth}
	Cabello, M. R.; \textbf{Angulo, L. D.}; Alvarez, J.; Bretones, A. R.; Gutierrez, G.; Garcia, S.\\
	A New efficient and stable 3D Conformal FDTD.\\
	{\it IEEE Microwave Component Letters}, {\bf 2016}, 26, 553-555.
\end{minipage}

\begin{minipage}{\textwidth}
	Gutierrez, G. G.; Mateos, D.; Cabello, M. R.; Pascual-Gil, E.; \textbf{Angulo, L. D.}; Garcia, S.\\
	On the design of Aircraft Electrical Structure Networks.\\
	\textit{IEEE Transacions on Electromagnetic Compatibility}, \textbf{2016}, 58, 401-408
\end{minipage}

Alvarez, J.; {\bf Angulo, L. D.}; Bretones, A. R.; Garcia, S.; M. de Jong van Coevorden, C.\\
Efficient Antenna Modeling by DGTD: Leap-frog discontinuous Galerkin timedomain method. \\
{\it IEEE Antennas and Propagation Magazine}, {\bf 2015}, 57, 95-106.

{\bf Angulo, L. D.}; Alvarez, J.; Pantoja, M.; Garcia, S.; Bretones, A. R.\\
Discontinuous Galerkin Time Domain Methods in Comp. Electrodynamics: State of the Art.\\
{\it Forum for Electromagnetic Res. Methods and App. Technologies (FERMAT)}, {\bf 2015}, 10.

{\bf Angulo, L. D.}; Alvarez, J.; Teixeira, F. L.; Pantoja, M.; Garcia, S.\\
A Nodal Hybrid Continuous-Discontinuous Galerkin Time Domain Method for Maxwell's Curl Equations.\\
{\it IEEE Transactions On Microwave Theory and Techniques}, {\bf 2015}, 63, 10, 3081-3093.

{\bf Angulo, L. D.}; Alvarez, J.; Pantoja, M.; Garcia, S.\\
A Nodal Space-Time Discontinuous Galerkin Method for the Maxwell Equations.\\
{\it IEEE Microwave Component Letters}, {\bf 2014}, 24, 827-829.

{\bf Angulo, L. D.}; Alvarez, J.; Teixeira, F. L.; Pantoja, M.; Garcia, S.\\
Causal-Path Local Time-Stepping in the Discontinuous Galerkin method for Maxwell's equations.\\
{\it Journal of Computational Physics}, {\bf 2014}, 256, 678 - 695.

Alvarez, J.; {\bf Angulo, L. D.}; Bretones, A. R.; Garcia, S. G.\\
An analysis of the Leap-Frog Discontinuous Galerkin method for Maxwell’s equations.\\
{\it IEEE Transactions on Microwave Theory and Techniques}, {\bf 2014}, 62, 197-207.

Alvarez, J.; Alonso-Rodriguez, J. M.; Carbajosa-Cobaleda, H.; Cabello, M. R.; {\bf Angulo, L. D.}; Gomez-Martin, R.; Garcia, S. G.\\
DGTD for a Class of Low-Observable Targets: A Comparison with MoM and (2,2) FDTD.\\
{\it IEEE Antennas and Wireless Propagation Letters}, {\bf 2014}, 13, 241-244.

Alvarez, J.; {\bf Angulo, L. D.}; Bretones, A.; Cabello, M. ; Garcia, S.\\
A Leap-Frog Discontinuous Galerkin Time-Domain Method for HIRF Assessment.\\
{\it Electromagnetic Compatibility, IEEE Transactions on}, {\bf 2013}, 55, 1250-1259.

Lin, H.; Pantoja, M. F.; {\bf Angulo, L. D.}; Alvarez, J.; Martin, R. G. ; Garcia, S. G.\\
FDTD Modeling of Graphene Devices Using Complex Conjugate Dispersion Material Model.\\
{\it Microwave and Wireless Components Letters, IEEE}, {\bf 2012}, 22, 612 -614

Alvarez, J.; {\bf Angulo, L. D.}; Rubio Bretones, A. ; Gonzalez Garcia, S.\\
3D discontinuous Galerkin time--domain method for anisotropic materials.\\
{\it Antennas and Wireless Propagation Letters, IEEE}, {\bf 2012}, 11, 1182-1185.

Alvarez, J.; {\bf Angulo, L. D.}; Rubio Bretones, A. ; Garcia, S.\\
A Spurious-Free Discontinuous Galerkin Time-Domain Method for the Accurate Modeling of Microwave Filters.\\
{\it Microwave Theory and Techniques, IEEE Transactions on}, {\bf 2012}, 60, 2359-2369.

{\bf Angulo, L. D.}; Alvarez, J.; Garcia, S.; Bretones, A. R. ; Martin, R. G.\\
Discontinuous Galerkin time-domain method for GPR simulation of conducting objects.\\
{\it Near Surface Geophysics}, {\bf 2011}, 9, 257-263

Alvarez, J.; {\bf Angulo, L. D.}; Fernandez Pantoja, M.; Rubio Bretones, A.; Garcia, S.\\
Source and Boundary Implementation in Vector and Scalar DGTD.\\
{\it Antennas and Propagation, IEEE Transactions on}, {\bf 2010}, 58, 1997 -2003.

Garcia, S. G.; Costen, F.; Fernandez Pantoja, M.; {\bf Angulo, L. D.}; Alvarez, J.\\
Efficient Excitation of Waveguides in Crank-Nicolson FDTD.\\
{\it Progress In Electromagnetics Research Letters}, {\bf 2010}, 17, 39-46.

Sanchez, C. C.; Garcia, S. G.; {\bf Angulo, L. D.}; Coevorden, C. M. D. J. V.; Bretones, A. R.\\
A divergence-free BEM method to model quasi-static currents: application to MRI coil design.\\
{\it Progress In Electromagnetics Research B}, {\bf 2010}, 20, 187-203.

{\bf Angulo, L. D.}; Garcia, S.; Pantoja, M.; Sanchez, C.; Martin, R.\\ 
Improving the SAR Distribution in Petri-Dish Cell Cultures.\\
{\it Journal of Electromagnetic Waves and Applications}, {\bf 2010}, 24, 815-826(12).


\section{\sc Selected Conference Contributions}

\textbf{As part of the organization:}\\[0.5cm]
\begin{minipage}{\textwidth}
	\textit{chairman} of several sessions during the \textit{EMC Europe}, \textbf{2019}, Barcelona.
\end{minipage}

\begin{minipage}{\textwidth}
	Organizer and chairman \textit{chairman} of session:\\
	Advances in Time Domain Numerical Techniques,\\
	\textit{IEEE-MTT-S Internacional Conference on Numerical Electromagnetic and Multiphysics Modeling and Optimization (NEMO)}, \textbf{2017}, Sevilla.
\end{minipage}

\begin{minipage}{\textwidth}
	Local organizing committee:\\
	\textit{Computational Electromagnetics for EMC (CEMEMC)}, \textbf{2013}, Granada.
\end{minipage}

\begin{minipage}{\textwidth}
		Local organizing committee:\\
	\textit{5th International Workshop on Advanced Ground Penetrating Radar (IWAGPR)}, \textbf{2009}, Granada
\end{minipage}

\textbf{As presenter:}\\[0.5cm]
\begin{minipage}{\textwidth}
	Hopf Solutions of Maxwell Equations: Analysis and Simulation by FDTD.\\
	A. J. M. Valverde; Garcia, S.; Cabello, M.; \textbf{Angulo, L. D.}; Bretones, A.; Alvarez, J.\\
	\textit{International Conference on Electromagnetics in Advanced Applications}, \textbf{2019}
\end{minipage}

\begin{minipage}{\textwidth}
	Application of Stochastic FDTD to thin-wire analysis.\\
	Garcia, S.; Cabello, M.; \textbf{Angulo, L. D.}; Bretones, A.; Alvarez, J.\\
	\textit{EMC Europe}, \textbf{2019}
\end{minipage}

\begin{minipage}{\textwidth}
	A Novel Subgriding Scheme for Arbitrarily Dispersive Thin-layer Modeling.\\
	Cabello, M.; \textbf{Angulo, L. D.}; Bretones, A.; Martin, R.; Garcia, S.; Alvarez, J.\\
	\textit{IEEE-MTT-S Internacional Conference on Numerical Electromagnetic and Multiphysics Modeling and Optimization (NEMO)}, \textbf{2017}
\end{minipage}

\begin{minipage}{\textwidth}
	A new FDTD subgridding boundary condition for FDTD subcell lossy thin-layer modeling Antennas and Propagation.\\
	Cabello, M.; \textbf{Angulo, L. D.}; Bretones, A.; Martin, R.; Garcia, S.; Alvarez, J.\\
	\textit{(APSURSI) IEEE International Symposium on Antennas and Propagation}, \textbf{2016}, 2031-2032
\end{minipage}

\begin{minipage}{\textwidth}
	A Novel Subgriding Scheme for Arbitrarily Dispersive Thin-layer Modeling.\\
	Cabello, M.; \textbf{Angulo, L. D.}; Bretones, A.; Martin, R.; Garcia, S.; Alvarez, J.\\
	\textit{IEEE-MTT-S Internacional Conference on Numerical Electromagnetic and Multiphysics Modeling and Optimization (NEMO)}, \textbf{2017}
\end{minipage}

\begin{minipage}{\textwidth}
	A new FDTD subgridding boundary condition for FDTD subcell lossy thin-layer modeling Antennas and Propagation.\\
	Cabello, M.; \textbf{Angulo, L. D.}; Bretones, A.; Martin, R.; Garcia, S.; Alvarez, J.\\
	\textit{(APSURSI) IEEE International Symposium on Antennas and Propagation}, \textbf{2016}, 2031-2032
\end{minipage}

\begin{minipage}{\textwidth}
{\bf Angulo, L. D.}; Alvarez, J.; Pantoja, M. F.; Garcia, S. G.; Bretones, A. R.\\
Recent Developments for Discontinuous {Galerkin} Time Domain Methods in Computational Electrodynamics\\
{\it X Iberian Meeting on Computational Electromagnetics}, {\bf 2015}.
\end{minipage}

Alvarez, J.; {\bf Angulo, L. D.}; Bretones, A. R.; Garcia, S. G.\\
A comparison of the FDTD and LFDG methods for the estimation of HIRF transfer functions.\\
{\it Proc. on Computational ElectroMagnetics And Electromagnetic Compatibility (CEMEMC13)}, {\bf 2013}.

Alvarez, J.; {\bf Angulo, L. D.}; Bandinelli, M.; Bruns, H.; Francavilla, M.; Garcia, S.; Guidi, R.; Gutierrez, G.; Jones, C.; Kunze, M.; Martinaud, J.; Munteanu, I.; Panitz, M.; Parmantier, J.; Pirinoli, P.; Reznicek, Z.; Salin, G.; Schroder, A.; Tobola, P.; Vipiana, F.\\
HIRF interaction with metallic aircrafts. A comparison between TD and FD methods.\\
{\it Electromagnetic Compatibility (EMC EUROPE), International Symposium on}, {\bf 2012}.

{\bf Angulo, L. D.}; Greco, S.; Ruiz-Cabello, M.; G. Garcia, S.; Sarto, M. S.\\
FDTD techniques to simulate composite air vehicles for EMC.\\
{\it AES Symposium}, Paris, {\bf 2012}.

\begin{minipage}{\textwidth}
Alvarez, J.; Gutierrez, G.;  {\bf Angulo, L. D.}; Lin, H.; R. Bretones, A.; G. Garcia, S.\\
Novel Time Domain FE/FD Solvers for EMC Assessment.\\
{\it VIII EIEC Encontro Iberico de Electromagnetismo computacional}, {\bf 2011}
\end{minipage}

\begin{minipage}{\textwidth}
Alvarez, J.; Garcia, S.;  {\bf Angulo, L. D.}; Bretones, A.\\
Computational Electromagnetic Tools for EMC in Aerospace. 
{\it Computational Electromagnetics, Piers Proceedings}, Marrakesh, {\bf 2011}
\end{minipage}

\begin{minipage}{\textwidth}
Alvarez, J.;  {\bf Angulo, L. D.}; Garcia, S. G.; Pantoja, M. F.; Bretones, A. R.\\
A Comparison Between Upwind/Centered Nodal/Vector Basis DGTD.\\
{\it IEEE International Symposium on Antennas and Propagation and CNC/USNC/URSI Radio Science Meeting}, {\bf 2010}
\end{minipage}

\begin{minipage}{\textwidth}
{\bf Angulo, L. D.}; Alvarez, J.; Bretones, A. R.; Garcia, S. G.\\
Time Domain Tools in EMC Assesment in Aeronautics.\\
{\it EMC Europe}, {\bf 2010}.
\end{minipage}

\begin{minipage}{\textwidth}
{\bf Angulo, L. D.}; Alvarez, J.; Garcia, S. G.; Pantoja, M. F.; Bretones, A. R.\\
CDGTD: A new reduced error method combining FETD and DGTD.\\
{\it IEEE International Symposium on Antennas and Propagation and CNC/USNC/URSI Radio Science Meeting}, {\bf 2010}
\end{minipage}

\begin{minipage}{\textwidth}
Bahl, R.;  {\bf Angulo, L. D.}; G. Garcia, S.; Bretones, A.; F. Pantoja, M.; Moreno de Jong van Coevorden, C.; Gomez Martin, R.\\
Numerical Dosimetry of Cell Cultures.\\
{\it Proceedings of the VI Iberian Meeting on Computational Electromagnetism}, {\bf 2008}
\end{minipage}


\end{document}




