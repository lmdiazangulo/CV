\documentclass[margin,line]{res}

\usepackage[utf8]{inputenc}
\usepackage[normalem]{ulem}
\usepackage{todonotes}
\usepackage[spanish]{babel}
\usepackage{url}

%\oddsidemargin -.5in
%\evensidemargin -.5in

%\textwidth=5.0in
\itemsep=0in
\parsep=0in
% if using pdflatex:
%\setlength{\pdfpagewidth}{\paperwidth}
%\setlength{\pdfpageheight}{\paperheight} 
\widowpenalties 1 10000

\newenvironment{list1}{
  \begin{list}{\ding{113}}{%
      \setlength{\itemsep}{0in}
      \setlength{\parsep}{0in} \setlength{\parskip}{0in}
      \setlength{\topsep}{0in} \setlength{\partopsep}{0in} 
      \setlength{\leftmargin}{0.17in}}}{\end{list}}
\newenvironment{list2}{
  \begin{list}{$\bullet$}{%
      \setlength{\itemsep}{0in}
      \setlength{\parsep}{0in} \setlength{\parskip}{0in}
      \setlength{\topsep}{0in} \setlength{\partopsep}{0in} 
      \setlength{\leftmargin}{0.2in}}}{\end{list}}

\begin{document}

\name{Luis Manuel Díaz Angulo \vspace*{.1in}}

\begin{resume}
\section{\sc Puesto actual}
 Profesor Titular en la Universidad de Granada (Área de Electromagnetismo).
 
\section{\sc Datos personales}
DNI: 74.933.051-V\\
Fecha y lugar de nacimiento: 08/08/1985, Barakaldo (Bizkaia).\\
Domicilio: Avenida de Murcia, 22, casa 2. 18012. Granada.\\
Estado civil: soltero.
 
\section{\sc Información de contacto}
\vspace{.05in}
\begin{tabular}{@{}p{3.1in}p{4in}}
Dpto. de Electromagnetismo y Física de la Materia  & {\it Trabajo:}  (+34) 667-87-69-42 \\            
Facultad de Ciencias& \\         
Universidad de Granada & \\
Granada, 18003, España            &{\it e-mail:}  LM@DiazAngulo.com
\end{tabular}

\section{\sc Destrezas} 
 {\bf Idiomas}
  \begin{list2}
   \item Español, lengua materna.
   \item Inglés, fluido (certificación B2).
  \end{list2}
  {\bf Ingeniería electrónica}
  \begin{list2} 
   \item Resolución de problemas de compatibilidad electromagnética. 
   \item Desarrollo de simuladores y métodos numéricos empleados en compatibilidad electromagnética, diseño de antenas, etc. 
   \item Simulación (uso de programas comerciales): Ansys HFSS, FEKO, CST Microwave Studio, OpenFoam (Advanced certification), GiD.
  \end{list2}
 {\bf Programación (Portfolio público en: \url{https://github.com/lmdiazangulo})}
 \begin{list2} 
  \item C++
  \item Python
 \end{list2}
 
\section{\sc Educación}
{\bf Universidad de Granada (UGR)}, Granada, España\\
\vspace*{-.1in}
\begin{itemize}
  \item Licenciado en Ingeniería Electrónica \hfill {\bf Oct. 2008 - Feb. 2015}
  \begin{list2}
    \item[-] Proyecto Fin de Carrera: {\it Discontinuous Galerkin Methods for the accurate modeling of Microwave Filters}.
    \item[-] Nota media de 7.123 sobre 10.
  \end{list2}
\vspace*{.1in}
\item Doctor por la Universidad de Granada \hfill {\bf Jul. 2010 - Nov. 2014}
 \begin{list2}
  \vspace*{.05in}
  
  \item[-] Título de la tesis: {\it Discontinuous Galerkin Methods applied to Computational Electromagnetics}
  \item[-] Calificación: Sobresaliente, con mención internacional.
  \item[-] Programa de doctorado de Física y Matemáticas (Fisymat).
  \item[-] Directores:  Dr. Salvador G. Garcia, Dr. Ing. Mario F. Pantoja, Dr. Ing. Jesus G. Alvarez.
 \end{list2}
\vspace*{.1in}
\item Licenciado en Física  \hfill {\raggedleft \bf Oct. 2003 - Jul. 2008}
 \begin{list2}
  \item[-] El año académico 2007/08 realicé una estancia ERASMUS en {\it The University of Manchester}, Reino Unido.
  \item[-] Las asignaturas optativas que cursé se centraban en el área de  electromagnetismo.
  \item[-] Nota media de 7.056 sobre 10. 
 \end{list2}
\vspace*{.1in}
 \item  Master en Física y Matemáticas (Fisymat) \hfill {\raggedleft \bf Oct. 2009 - Dic. 2010}
 \begin{list2}
  \item[-] Trabajo final de máster (Calif. Matrícula de honor): {\it Discontinuous Galerkin Time
  Domain Method applied to Maxwell’s Equations: Implementation, validation and application cases}
 \end{list2}
\vspace*{.1in}
 \item Máster Univ. Prof. de ESO, Bach., FP y EOI \hfill {\raggedleft \bf Oct. 2014 - Sep. 2015}
 \begin{list2}
  \item[-] Título final de máster: \textit{Calendarios}.
 \end{list2}
\end{itemize}

\begin{minipage}{\textwidth}
{\bf Universidad Politécnica de Cataluña (UPC)}, Barcelona, España\\
\vspace*{-.1in}
\begin{itemize}
\item Escuela de verano en métodos discontinuos de Galerkin  \hfill {\bf Jun. 2012}
\item Curso corto en tecnología y aplicaciones de los terahercios \hfill {\bf May. 2009}
\end{itemize}
\end{minipage}

\section{\sc Experiencia académica}
{\bf Universidad de Granada}, Granada, España\\
{\em Prof. sustituto interino/Prof. ayudante doctor} \hfill {\bf Mar. 2012 - Actualidad}\\
Más de mil horas acreditadas de docencia en grado (Ingeniería Electrónica, Física, Química) y posgrado (Máster en Física y Matemáticas):

\begin{minipage}{\textwidth}
	\section{\sc Publicaciones}
    Más de 20 artículos publicados en revistas indexadas, de ámbito internacional; 12 de ellos situados en el primer tercil.
\end{minipage}


\section{\sc Participación en conferencias}
\begin{minipage}{\textwidth}
  \begin{list2}
    \item Responsable de las V Jornadas Españolas de EMC (\url{http://jornadasemc.es/}).
    \item Más de 25 contribuciones a conferencias de ámbito internacional.
  \end{list2} 
\end{minipage}

\section{\sc Patentes y propiedad intelectual}
\begin{minipage}{\textwidth}
  {\em SEMBA-UGRFDTD}\\
  SEMBA-UGRFDTD es un programa para la simulación de problemas electromagnéticos, mediante el método de Diferencias Finitas en el Dominio del Tiempo (FDTD), con énfasis en problemas de compatibilidad electromagnética. Actualmente se ha licenciado su uso a Airbus Group (Europa), INTA (España), GiD (España), y la National University of Defence and Technology (República Popular de China).
\end{minipage}



\begin{minipage}{\textwidth}
	\section{\sc Proyectos y contratos, como investigador principal}
   {\em COMPEE (UE - FEDER)} \hfill {\bf Sep. 2021 - Sep. 2022}\\
   {\em NiTest (Airbus)} \hfill {\bf Sep. 2018 - Sep. 2020}\\
   {\em Exp. didácticos para la enseñanza del electromagnetismo.} \hfill {\bf Feb. 2017 - Feb. 2019}\\
   {\em Soporte y mantenimiento de SEMBA (NUDT)} \hfill {\bf Ene. 2017 - Dic. 2018}\\
   {\em Integración del mallador ZMesher en GiD (CIMNE)} \hfill {\bf Ene. 2017 - Dic. 2017}\\  
\end{minipage}

\begin{minipage}{\textwidth}
	\section{\sc Proyectos y contratos, como investigador}
  {\em NT-FDTD (Huawei)} \hfill {\bf Sep. 2018 - Sep. 2021}\\
	{\em Alhambra LFT (Airbus)} \hfill {\bf Sep. 2018 - Sep. 2021}\\
	{\em COST ACCREDIT (UE - COST)} \hfill {\bf Ene. 2017 - Dic. 2019}\\
  {\em PRACE SREDIT (UE - PRACE)} \hfill {\bf Ene. 2017 - Dic. 2019}\\
  {\em UAVE-3 (Plan Nacional de Investigación)} \hfill {\bf Ene. 2017 - Dic. 2019}\\
  {\em UAVEMI  (Plan Nacional de Investigación)} \hfill {\bf Ene. 2014 - Dic. 2016}\\
  {\em MORFEO (Airbus)} \hfill {\bf Sep. 2014 - Dic. 2015}\\
  {\em MANIAS (Airbus)} \hfill {\bf Dic. 2014 - Dic. 2015}\\
  {\em A-UGRFDTD (Airbus)} \hfill {\bf Sep. 2012 - Sep. 2015}\\
  {\em TERALAB (J. Andalucía - Proy. Excelencia)} \hfill {\bf Nov. 2009 - Nov. 2013}\\
  {\em HIRF: SE (UE - Programa Marco)} \hfill {\bf Dic. 2008 - Jun. 2013}\\  
  {\em TERASENSE (Plan Nacional de Investigación)} \hfill {\bf Oct. 2008 - Mar. 2012}\\
  {\em Efectos biológicos de la Radiofrecuencia (Min. Defensa)} \hfill {\bf Jul. 2008 - Dic. 2008}\\
  {\em Téc. GPR para valoración arquitectónica (J. Andalucía)} \hfill {\bf Jul. 2008 - Mar. 2010}\\
  {\em Téc. de optimización para antenas UWB (Min. de Defensa)} \hfill {\bf Oct. 2007 - Dic. 2010}\\
\end{minipage}

\end{resume}
\end{document}