\documentclass[a4paper,margin,line]{res}

\usepackage[utf8]{inputenc}
\usepackage[normalem]{ulem}
\usepackage{todonotes}


%\oddsidemargin -.5in
%\evensidemargin -.5in

%\textwidth=5.0in
\itemsep=0in
\parsep=0in
% if using pdflatex:
%\setlength{\pdfpagewidth}{\paperwidth}
%\setlength{\pdfpageheight}{\paperheight} 
\widowpenalties 1 10000

\newenvironment{list1}{
  \begin{list}{\ding{113}}{%
      \setlength{\itemsep}{0in}
      \setlength{\parsep}{0in} \setlength{\parskip}{0in}
      \setlength{\topsep}{0in} \setlength{\partopsep}{0in} 
      \setlength{\leftmargin}{0.17in}}}{\end{list}}
\newenvironment{list2}{
  \begin{list}{$\bullet$}{%
      \setlength{\itemsep}{0in}
      \setlength{\parsep}{0in} \setlength{\parskip}{0in}
      \setlength{\topsep}{0in} \setlength{\partopsep}{0in} 
      \setlength{\leftmargin}{0.2in}}}{\end{list}}

\begin{document}

\name{Luis Manuel Díaz Angulo \vspace*{.1in}}

\begin{resume}
\section{\sc Ocupación actual}
 Profesor Ayudante Doctor
 
\section{\sc Datos personales}
DNI: 74.933.051-V\\
Fecha de nacimiento: 08/08/1985\\
Lugar de nacimiento: Barakaldo (Vizcaya)\\
Estado civil: soltero
 
\section{\sc Información de contacto}
\vspace{.05in}
\begin{tabular}{@{}p{3.1in}p{4in}}
Dpto. de Electromagnetismo y Física de la Materia  & {\it Trabajo:}  (+34) 667-87-69-42 \\            
Facultad de Ciencias             & \\         
Universidad de Granada            & \\
Granada, 18003, España            &{\it e-mail:}  LM@DiazAngulo.com
\end{tabular}


%\section{\sc Research Interests}
%Computational Electromagnetics,5
%Electromagnetic Phenomena, 
%Numerical methods, 
%Galerkin methods,
%Plasma technology,
%Terahertz technology,
%Electromagnetic Compatibility,
%High Performance Computing.

\section{\sc Destrezas principales} 
 {\bf Idiomas}
  \begin{list2}
   \item Español, lengua materna.
   \item Inglés, fluido (certificación B2).
  \end{list2}
 {\bf Informática}
 \begin{list2} 
  \item Programación:  C/C++, Python, Matlab, Tcl/TK, Fortran.
  \item Simulación/CAD: Ansys HFSS, FEKO, CST Microwave Studio, OpenFoam (Advanced certification), GiD.
  \end{list2}

\section{\sc Educación}
{\bf Universidad de Granada (UGR)}, Granada, España\\
\vspace*{-.1in}
\begin{itemize}
  \item Licenciado en Ingeniería Electrónica \hfill {\bf Oct. 2008 - Feb. 2015}
  \begin{list2}
    \item[-] Proyecto Fin de Carrera: {\it Discontinuous Galerkin Methods for the accurate modeling of Microwave Filters}.
    \item[-] Nota media de 7.123 sobre 10.
  \end{list2}
\vspace*{.1in}
\item Doctor por la Universidad de Granada \hfill {\bf Jul. 2010 - Nov. 2014}
 \begin{list2}
  \vspace*{.05in}
  
  \item[-] Título de la tesis: {\it Discontinuous Galerkin Methods applied to Computational Electromagnetics}
  \item[-] Calificación: Sobresaliente, con mención internacional.
  \item[-] Programa de doctorado de Física y Matemáticas (Fisymat).
  \item[-] Directores:  Dr. Salvador G. Garcia, Dr. Ing. Mario F. Pantoja, Dr. Ing. Jesus G. Alvarez.
 \end{list2}
\vspace*{.1in}
\item Licenciado en Física  \hfill {\raggedleft \bf Oct. 2003 - Jul. 2008}
 \begin{list2}
  \item[-] El año académico 2007/08 realicé una estancia ERASMUS en {\it The University of Manchester}, Reino Unido.
  \item[-] Las asignaturas optativas que cursé se centraban en el área de  electromagnetismo.
  \item[-] Nota media de 7.056 sobre 10. 
 \end{list2}
\vspace*{.1in}
 \item  Master en Física y Matemáticas (Fisymat) \hfill {\raggedleft \bf Oct. 2009 - Dic. 2010}
 \begin{list2}
  \item[-] Trabajo final de master (Calif. Matrícula de honor): {\it Discontinuous Galerkin Time
  Domain Method applied to Maxwell’s Equations: Implementation, validation and application cases}
 \end{list2}
\vspace*{.1in}
 \item Máster Univ. Prof. de ESO, Bach., FP y EOI \hfill {\raggedleft \bf Oct. 2014 - Sep. 2015}
 \begin{list2}
  \item[-] Título de proyecto Fin de Máster: \textit{Calendarios}.
 \end{list2}
\end{itemize}

\begin{minipage}{\textwidth}
{\bf Universidad Politécnica de Cataluña (UPC)}, Barcelona, España\\
\vspace*{-.1in}
\begin{itemize}
\item Escuela de verano en métodos discontinuos de Galerkin  \hfill {\bf Junio 2012}
\item Curso corto en tecnología y aplicaciones de los terahercios \hfill {\bf Mayo 2009}
\end{itemize}
\end{minipage}

\section{\sc Experiencia académica}
{\bf Universidad de Granada}, Granada, España\\
{\em Prof. sustituto interino/Prof. ayudante doctor} \hfill {\bf Nov. 2015 - Actualidad}\\
Docencia en grado y posgrado:
\begin{list2}
  \item \textit{Métodos Computacionales en Física No Lineal} en el Máster de Física y Matemáticas (FisyMat).
  \item \textit{Física II} en el primer curso del grado en Ingeniería Química.
  \item \textit{Física II} en el primer curso del grado en Química.
  \item \textit{Electromagnetismo} en el primer curso del grado en Ingeniería Electrónica Industrial.
  \item \textit{Electromagnetismo} en el tercer curso del grado en Física.
\end{list2}

Dirección de tesis doctorales:
\begin{list2}
  \item Subcell FDTD Techniques for electromagnetic compatibility assesment in Aeronautics.\\
  Realizado por: Miguel David Ruiz-Cabello Nuñez. \\Dirigido por García, S.G. y \textbf{Angulo, L. D.}\\
  \textit{Presentación prevista en septiembre de 2017.}
\end{list2}

Dirección de Trabajos de Fin de Máster (TFM):
\begin{list2}
  \item Uso de Realidad Virtual y Laboratorios Virtuales para la Enseñanza-Aprendizaje de las Ciencias Experimentales en Secundaria.\\
  Realizado por: Joaquín Castellano Simón. Dirigido por Castillo-Rosúa, F. J. y \textbf{Angulo, L. D.}\\
  \textit{Presentación prevista en septiembre de 2017.}
\end{list2}

Dirección de Trabajos de Fin de Grado (TFG):
\begin{list2}
  \item Estudio de Efectividades de Apantallamiento en materiales compuestos.\\
  Realizado por: Alejandro Ramos Lora. Dirigido por \textbf{Angulo, L. D.} y García, S. G.\\
  Línea de Métodos \textit{Computacionales en Electromagnetismo}.\\
  \textit{Presentación prevista en septiembre de 2017.}
\end{list2}

{\bf Universidad de Granada}, Granada, España\\
{\em Profesor} \hfill {\bf Mar. 2012 - Sep. 2014}\\
Responsabilidad compartida en clases de teoría, sesiones de laboratorio, realización de exámenes, trabajos y notas.
\vspace*{.05in}  
\begin{list2}
 \item \textit{Electromagnetismo} en el primer curso del grado en Ingeniería Electrónica Industrial.
 \item \textit{Electromagnetismo en} el tercer curso del grado en Física.
\end{list2}

\section{\sc Estancias}
\begin{minipage}{\textwidth}
	{\bf Central China Normal University}, Wuhan, República Popular de China\\
	{\em Estancias de investigación/intercambio} \hfill {\bf Julio 2019}
	\begin{itemize}
		\item Estancia de un mes para intercambios científicos sobre métodos numéricos.
		\item Impartí cursos de formación para el uso de la herramienta computacional SEMBA-UGRFDTD desarrollada por el GEG.
	\end{itemize}
\end{minipage}

\begin{minipage}{\textwidth}
	{\bf National University of Defence Technology}, Changsha, República Popular de China\\
	{\em Estancias de investigación/intercambio} \hfill {\bf 2015, 2017, 2019}
	\begin{itemize}
		\item Tres estancias de diez días para intercambios científicos sobre métodos numéricos.
	\end{itemize}
\end{minipage}

\begin{minipage}{\textwidth}
	{\bf The Ohio State University - ElectroScience Laboratory}, Columbus, Ohio USA\\
	{\em Estancia de investigación} \hfill {\bf Ene. 2013 - Abr. 2013}
	\begin{itemize}
		\item Tres meses en la Universidad Estatal de Ohio, bajo la supervisión del catedrático Fernando Lisboa Teixeira.
		\item Durante este periodo, desarrollé una técnica nueva para mejorar el rendimiento de las simulaciones de onda completa que ha sido publicado en la revista científica: {\it Journal of Computational Physics}.
	\end{itemize}
\end{minipage}

{\bf Inst. National de Recherche en Informatique et en Automatique (INRIA)}, Niza, Francia\\
{\em Estancia de investigación} \hfill {\bf Sep. 2012 - Dic. 2012}\\
\begin{itemize}
 \item Tres meses investigando sobre métodos discontinuos de Galerkin en el espacio-tiempo (STDG) bajo la supervisión del Dr. Stephane Lanteri.
 \item Impartí dos sesiones sobre {\it Modelado de materiales avanzados} y {\it STDG}.
\end{itemize}

{\bf Int. Center for Numerical Methods in Engineering (CIMNE)}, Barcelona, España\\
{\em Estancia de investigación} \hfill {\bf Mayo 2010}\\
\begin{itemize}
 \item Dos semanas participando en el desarrollo de una interfaz gráfica para la herramienta UGRFDTD.
\end{itemize}

{\bf Dassault Systems}, Saint Cloud, París, Francia\\
\vspace{-.3cm}
{\em Instructor de taller} \hfill {\bf Marzo 2012}\\
\begin{itemize}
 \item Como parte del proyecto HIRF-SE, instruí a otros participantes del proyecto en el uso del software para simulaciones FDTD desarrollado por la Universidad de Granada (UGRFDTD).
\end{itemize}

\begin{minipage}{\textwidth}
\section{\sc Revisión de artículos}
Realizo regularmente revisiones de artículos para las siguientes revistas científicas:
\begin{itemize}
 \item Journal Of Computational Physics (Factor de impacto en 2012: 2.84)
 \item IEEE Transactions On Microwave Theory and Techniques (Factor de impacto en 2012: 2.23).
 \item IEEE Antennas and Wireless Propagation Letters (Factor de impacto 2012: 1.67).
 \item Progress In Electromagnetic Research - Journal of Electromagnetic Waves and Applications (PIER-JEMWA) (Factor de impacto en 2012: 1.60).
 \item Near Surface Geophysics (NSG) (Factor de impacto en 2012: 1.12).
\end{itemize}
\end{minipage}

\begin{minipage}{\textwidth}
	\section{\sc Publicaciones}
	Modeling and Measuring the Shielding Effectiveness of Carbon Fiber Composites.\\
	 \textbf{Angulo, L. D.}; P. Gómez; B. Plaza; D. Poyatos; Cabello, M. R.; Bocanegra, D. E.; Garcia, S. G.\\
	{\it IEEE Journal on Multiscale and Multiphysics Computational Techniques}, \textbf{2019}, 67, 207-213.
\end{minipage}

\begin{minipage}{\textwidth}
    Application of Stochastic FDTD to Holland's Thin-Wire Method.\\
	Garcia, S. G.; Cabello, M. R.; \textbf{Angulo, L. D.}; Bretones, A. R.; M. G. Atienza; Pascual-Gil, E.; \\
	{\it IEEE Antennas and Wireless Propagation Letters}, \textbf{2019}, 18, 2046-2050.
\end{minipage}

\begin{minipage}{\textwidth}
	A New Conformal FDTD for lossy thin panels.\\
	Cabello, M. R.; \textbf{Angulo, L. D.}; Alvarez, J.;  Bretones, A. R.;  Garcia, S. G.\\
	{\it IEEE Transactions on Antennas and Propagation}, \textbf{2019}, 67, 7433-7439.
\end{minipage}

\begin{minipage}{\textwidth}
  SIVA UAV: A Case Study for the EMC Analysis of Composite Air Vehicles.\\
  Cabello, M. R.; Fernández, S.; Pous, M.; Pascual-Gil, E.; \textbf{Angulo, L. D.}; (17 autores más)\\
  {\it IEEE Transactions on Electromagnetic Compatibility}, \textbf{2017}, 59, 1103-1113.
\end{minipage}

\begin{minipage}{\textwidth}
 Multiresolution Time-Domain Analysis of Multiconductor Transmission Lines Terminated in Linear Loads.\\
  Tong, Z.; Sun, L.; Li, Y.; \textbf{Angulo, L. D.}; Garcia, S. G.; Luo, J.\\
  {\it Mathematical Problems in Engineering}, {\bf 2017}, 15 páginas.
\end{minipage}

\begin{minipage}{\textwidth}
  A hybrid Crank-Nicolson FDTD subgridding boundary condition for lossy thin-layer modeling IEEE Transactions on Microwave Theory and Techniques.\\
  Cabello, M. R.; \textbf{Angulo, L. D.}; Alvarez, J.; Flintoft, I.; (4 autores más)\\
  {\it IEEE Transactions on Microwave Theory and Techniques}, {\bf 2017}, 65, 1397-1406.
\end{minipage}

\begin{minipage}{\textwidth}
  Cabello, M. R.; \textbf{Angulo, L. D.}; Alvarez, J.; Bretones, A. R.; Gutierrez, G.; Garcia, S.\\
  A New efficient and stable 3D Conformal FDTD.\\
  {\it IEEE Microwave Component Letters}, {\bf 2016}, 26, 553-555.
\end{minipage}

\begin{minipage}{\textwidth}
  Gutierrez, G. G.; Mateos, D.; Cabello, M. R.; Pascual-Gil, E.; \textbf{Angulo, L. D.}; Garcia, S.\\
  On the design of Aircraft Electrical Structure Networks.\\
  \textit{IEEE Transacions on Electromagnetic Compatibility}, \textbf{2016}, 58, 401-408
\end{minipage}

Alvarez, J.; {\bf Angulo, L. D.}; Bretones, A. R.; Garcia, S.; M. de Jong van Coevorden, C.\\
Efficient Antenna Modeling by DGTD: Leap-frog discontinuous Galerkin timedomain method. \\
{\it IEEE Antennas and Propagation Magazine}, {\bf 2015}, 57, 95-106.

{\bf Angulo, L. D.}; Alvarez, J.; Pantoja, M.; Garcia, S.; Bretones, A. R.\\
Discontinuous Galerkin Time Domain Methods in Comp. Electrodynamics: State of the Art.\\
{\it Forum for Electromagnetic Res. Methods and App. Technologies (FERMAT)}, {\bf 2015}, 10.

{\bf Angulo, L. D.}; Alvarez, J.; Teixeira, F. L.; Pantoja, M.; Garcia, S.\\
A Nodal Hybrid Continuous-Discontinuous Galerkin Time Domain Method for Maxwell's Curl Equations.\\
{\it IEEE Transactions On Microwave Theory and Techniques}, {\bf 2015}, 63, 10, 3081-3093.

{\bf Angulo, L. D.}; Alvarez, J.; Pantoja, M.; Garcia, S.\\
A Nodal Space-Time Discontinuous Galerkin Method for the Maxwell Equations.\\
{\it IEEE Microwave Component Letters}, {\bf 2014}, 24, 827-829.

{\bf Angulo, L. D.}; Alvarez, J.; Teixeira, F. L.; Pantoja, M.; Garcia, S.\\
Causal-Path Local Time-Stepping in the Discontinuous Galerkin method for Maxwell's equations.\\
{\it Journal of Computational Physics}, {\bf 2014}, 256, 678 - 695.

Alvarez, J.; {\bf Angulo, L. D.}; Bretones, A. R.; Garcia, S. G.\\
An analysis of the Leap-Frog Discontinuous Galerkin method for Maxwell’s equations.\\
{\it IEEE Transactions on Microwave Theory and Techniques}, {\bf 2014}, Accepted.

Alvarez, J.; Alonso-Rodriguez, J. M.; Carbajosa-Cobaleda, H.; Cabello, M. R.; {\bf Angulo, L. D.}; Gomez-Martin, R.; Garcia, S. G.\\
DGTD for a Class of Low-Observable Targets: A Comparison with MoM and (2,2) FDTD.\\
{\it IEEE Antennas and Wireless Propagation Letters}, {\bf 2014}, 13, 241-244.

Alvarez, J.; {\bf Angulo, L. D.}; Bretones, A.; Cabello, M. ; Garcia, S.\\
A Leap-Frog Discontinuous Galerkin Time-Domain Method for HIRF Assessment.\\
{\it Electromagnetic Compatibility, IEEE Transactions on}, {\bf 2013}, 55, 1250-1259.

Lin, H.; Pantoja, M. F.; {\bf Angulo, L. D.}; Alvarez, J.; Martin, R. G. ; Garcia, S. G.\\
FDTD Modeling of Graphene Devices Using Complex Conjugate Dispersion Material Model.\\
{\it Microwave and Wireless Components Letters, IEEE}, {\bf 2012}, 22, 612 -614

Alvarez, J.; {\bf Angulo, L. D.}; Rubio Bretones, A. ; Gonzalez Garcia, S.\\
3D discontinuous Galerkin time--domain method for anisotropic materials.\\
{\it Antennas and Wireless Propagation Letters, IEEE}, {\bf 2012}, 11, 1182-1185.

Alvarez, J.; {\bf Angulo, L. D.}; Rubio Bretones, A. ; Garcia, S.\\
A Spurious-Free Discontinuous Galerkin Time-Domain Method for the Accurate Modeling of Microwave Filters.\\
{\it Microwave Theory and Techniques, IEEE Transactions on}, {\bf 2012}, 60, 2359-2369.

{\bf Angulo, L. D.}; Alvarez, J.; Garcia, S.; Bretones, A. R. ; Martin, R. G.\\
Discontinuous Galerkin time-domain method for GPR simulation of conducting objects.\\
{\it Near Surface Geophysics}, {\bf 2011}, 9, 257-263

Alvarez, J.; {\bf Angulo, L. D.}; Fernandez Pantoja, M.; Rubio Bretones, A.; Garcia, S.\\
Source and Boundary Implementation in Vector and Scalar DGTD.\\
{\it Antennas and Propagation, IEEE Transactions on}, {\bf 2010}, 58, 1997 -2003.

Garcia, S. G.; Costen, F.; Fernandez Pantoja, M.; {\bf Angulo, L. D.}; Alvarez, J.\\
Efficient Excitation of Waveguides in Crank-Nicolson FDTD.\\
{\it Progress In Electromagnetics Research Letters}, {\bf 2010}, 17, 39-46.

Sanchez, C. C.; Garcia, S. G.; {\bf Angulo, L. D.}; Coevorden, C. M. D. J. V.; Bretones, A. R.\\
A divergence-free BEM method to model quasi-static currents: application to MRI coil design.\\
{\it Progress In Electromagnetics Research B}, {\bf 2010}, 20, 187-203.

{\bf Angulo, L. D.}; Garcia, S.; Pantoja, M.; Sanchez, C.; Martin, R.\\ 
Improving the SAR Distribution in Petri-Dish Cell Cultures.\\
{\it Journal of Electromagnetic Waves and Applications}, {\bf 2010}, 24, 815-826(12).

\section{\sc Organización de conferencias}
\begin{minipage}{\textwidth}
  Organizador y \textit{chairman} de la sesión:\\
  Advances in Time Domain Numerical Techniques, de la\\
  \textit{IEEE-MTT-S Internacional Conference on Numerical Electromagnetic and Multiphysics Modeling and Optimization (NEMO)}, \textbf{2017}, Sevilla.
\end{minipage}

\begin{minipage}{\textwidth}
  Participación en el comité organizador local del:\\
  \textit{Computational Electromagnetics for EMC (CEMEMC)}, \textbf{2013}, Granada.
\end{minipage}

\begin{minipage}{\textwidth}
  Participación en el comité organizador local del:\\
  \textit{5th International Workshop on Advanced Ground Penetrating Radar (IWAGPR)}, \textbf{2009}, Granada
\end{minipage}

\section{\sc Selección de \ \ Participaciones \ \  en Conferencias}
\begin{minipage}{\textwidth}
	Hopf Solutions of Maxwell Equations: Analysis and Simulation by FDTD.\\
	A. J. M. Valverde; Garcia, S.; Cabello, M.; \textbf{Angulo, L. D.}; Bretones, A.; Alvarez, J.\\
	\textit{International Conference on Electromagnetics in Advanced Applications}, \textbf{2019}
\end{minipage}

\begin{minipage}{\textwidth}
	Application of Stochastic FDTD to thin-wire analysis.\\
	Garcia, S.; Cabello, M.; \textbf{Angulo, L. D.}; Bretones, A.; Alvarez, J.\\
	\textit{EMC Europe}, \textbf{2019}
\end{minipage}

\begin{minipage}{\textwidth}
  A Novel Subgriding Scheme for Arbitrarily Dispersive Thin-layer Modeling.\\
  Cabello, M.; \textbf{Angulo, L. D.}; Bretones, A.; Martin, R.; Garcia, S.; Alvarez, J.\\
  \textit{IEEE-MTT-S Internacional Conference on Numerical Electromagnetic and Multiphysics Modeling and Optimization (NEMO)}, \textbf{2017}
\end{minipage}

\begin{minipage}{\textwidth}
A new FDTD subgridding boundary condition for FDTD subcell lossy thin-layer modeling Antennas and Propagation.\\
Cabello, M.; \textbf{Angulo, L. D.}; Bretones, A.; Martin, R.; Garcia, S.; Alvarez, J.\\
\textit{(APSURSI) IEEE International Symposium on Antennas and Propagation}, \textbf{2016}, 2031-2032
\end{minipage}

\begin{minipage}{\textwidth}
{\bf Angulo, L. D.}; Alvarez, J.; Pantoja, M. F.; Garcia, S. G.; Bretones, A. R.\\
Recent Developments for Discontinuous {Galerkin} Time Domain Methods in Computational Electrodynamics.\\
{\it X Iberian Meeting on Computational Electromagnetics}, {\bf 2015}.
\end{minipage}

Alvarez, J.; {\bf Angulo, L. D.}; Bretones, A. R.; Garcia, S. G.\\
A comparison of the FDTD and LFDG methods for the estimation of HIRF transfer functions.\\
{\it Proc. on Computational ElectroMagnetics And Electromagnetic Compatibility (CEMEMC13)}, {\bf 2013}.

Alvarez, J.; {\bf Angulo, L. D.}; Bandinelli, M.; Bruns, H.; Francavilla, M.; Garcia, S.; Guidi, R.; Gutierrez, G.; Jones, C.; Kunze, M.; Martinaud, J.; Munteanu, I.; Panitz, M.; Parmantier, J.; Pirinoli, P.; Reznicek, Z.; Salin, G.; Schroder, A.; Tobola, P.; Vipiana, F.\\
HIRF interaction with metallic aircrafts. A comparison between TD and FD methods.\\
{\it Electromagnetic Compatibility (EMC EUROPE), International Symposium on}, {\bf 2012}.

{\bf Angulo, L. D.}; Greco, S.; Ruiz-Cabello, M.; G. Garcia, S.; Sarto, M. S.\\
FDTD techniques to simulate composite air vehicles for EMC.\\
{\it AES Symposium}, Paris, {\bf 2012}.

\begin{minipage}{\textwidth}
Alvarez, J.; Gutierrez, G.;  {\bf Angulo, L. D.}; Lin, H.; R. Bretones, A.; G. Garcia, S.\\
Novel Time Domain FE/FD Solvers for EMC Assessment.\\
{\it VIII EIEC Encontro Iberico de Electromagnetismo computacional}, {\bf 2011}
\end{minipage}

\begin{minipage}{\textwidth}
Alvarez, J.; Garcia, S.;  {\bf Angulo, L. D.}; Bretones, A.\\
Computational Electromagnetic Tools for EMC in Aerospace. 
{\it Computational Electromagnetics, Piers Proceedings}, Marrakesh, {\bf 2011}
\end{minipage}

\begin{minipage}{\textwidth}
Alvarez, J.;  {\bf Angulo, L. D.}; Garcia, S. G.; Pantoja, M. F.; Bretones, A. R.\\
A Comparison Between Upwind/Centered Nodal/Vector Basis DGTD.\\
{\it IEEE International Symposium on Antennas and Propagation and CNC/USNC/URSI Radio Science Meeting}, {\bf 2010}
\end{minipage}

\begin{minipage}{\textwidth}
{\bf Angulo, L. D.}; Alvarez, J.; Bretones, A. R.; Garcia, S. G.\\
Time Domain Tools in EMC Assesment in Aeronautics.\\
{\it EMC Europe}, {\bf 2010}.
\end{minipage}

\begin{minipage}{\textwidth}
{\bf Angulo, L. D.}; Alvarez, J.; Garcia, S. G.; Pantoja, M. F.; Bretones, A. R.\\
CDGTD: A new reduced error method combining FETD and DGTD.\\
{\it IEEE International Symposium on Antennas and Propagation and CNC/USNC/URSI Radio Science Meeting}, {\bf 2010}
\end{minipage}

\begin{minipage}{\textwidth}
Bahl, R.;  {\bf Angulo, L. D.}; G. Garcia, S.; Bretones, A.; F. Pantoja, M.; Moreno de Jong van Coevorden, C.; Gomez Martin, R.\\
Numerical Dosimetry of Cell Cultures.\\
{\it Proceedings of the VI Iberian Meeting on Computational Electromagnetism}, {\bf 2008}
\end{minipage}

\section{\sc Patentes y propiedad intelectual}
\begin{minipage}{\textwidth}
  {\em SEMBA-UGRFDTD}\\
  SEMBA-UGRFDTD es un programa para la simulación de problemas electromagnéticos, mediante el método de Diferencias Finitas en el Dominio del Tiempo (FDTD), con énfasis en problemas de compatibilidad electromagnética. Actualmente se ha licenciado su uso a Airbus Group (Europa), INTA (España), GiD (España), y la National University of Defence and Technology (República Popular de China.
\end{minipage}

\begin{minipage}{\textwidth}
 {\em OpenSEMBA}\\
 OpenSEMBA es una librería, liberada como software libre (www.sembahome.org/opensemba), que incluye herramientas para la gestión de la información necesaria para la definición de problemas de electromagnetismo computacional. También incluye la infraestructura numérica necesaria para la creación de programas de simulación basados en el Método de Elementos Finitos.
\end{minipage}

\section{\sc Coordinación de proyectos y contratos}
\begin{minipage}{\textwidth}
	{\em NiTest} \hfill {\bf Sep. 2018 - Sep. 2020}\\
	Como investigador principal, en este proyecto se persigue mejorar el proceso industrial de Airbus. Para ello se evaluarán y desarrollaran nuevas técnicas no intrusivas en las etapas de testeo que aseguran la robustez de una aeronave al impacto de rayos. El proyecto está financiado integramente por Airbus.
\end{minipage}	

%\begin{minipage}{\textwidth}
%  {\em COBERT} \hfill {\bf (\textit{Est}.) Julio 2017 - Julio 2018}\\
%  Este proyecto será una colaboración con la Universidad de las Fuerzas Armadas (ESPE) de Ecuador. El objetivo es el desarrollo de una tecnología para la comunicación entre las fuerzas militares en zonas de muy baja cobertura como son las selváticas. Para ello se estudiara, experimental y teóricamente, la propagación de señales en este tipo de entornos.
%\end{minipage}

\begin{minipage}{\textwidth}
  {\em Desarrollo de experimentos didácticos de bajo coste para la enseñanza del electromagnetismo.} \hfill {\bf Feb. 2017 - Feb. 2019}\\
  Este proyecto busca la mejora del proceso enseñanza-aprendizaje en las clases de teoría y problemas, principalmente en las asignaturas de Física II del Grado en Química y del Grado en Ingeniería Química. Para ello se desarrollarán distintas demostraciones experimentales en el aula (de cátedra) como refuerzo a los contenidos teórico-prácticos. Un elemento característico y común de todos estos experimentos será su bajo coste económico, típicamente empleando elementos cotidianos. De cara a evaluar el efecto de la inclusión de estas experiencias, se desarrollará un estudio de asistencia y se evaluará su efecto en las calificaciones medias finales obtenidas por los estudiantes.
\end{minipage}

\begin{minipage}{\textwidth}
  {\em Soporte y mantenimiento de SEMBA (NUDT)} \hfill {\bf Ene. 2017 - Dic. 2018}\\
  Este contrato se ha suscrito entre la UGR y la National University of Defence and Technology (NUDT) de la República Popular de China. Se incluye la venta de una licencia del software SEMBA-UGRFDTD, soporte técnico, formación y mantenimiento. También incluye una formación presencial, que será impartida por la UGR, para la simulación de dispositivos en el contexto de compatibilidad electromagnética.
\end{minipage}

\begin{minipage}{\textwidth}
  {\em Integración del mallador ZMesher en GiD} \hfill {\bf Ene. 2017 - Dic. 2017}\\
  En este contrato se han vendido los derechos de explotación, no exclusivos, del mallador ZMesher, desarrollado por el GEG-UGR, al \textit{Centre Internacional de Mètodes Numèrics a l'Enginyeria} (CIMNE) para su integración en el programa GiD. Esta concesión implica el soporte durante un año para la integración del mallador.
\end{minipage}

\begin{minipage}{\textwidth}
	\section{\sc Participación en proyectos y contratos}
	{\em Alhambra LFT} \hfill {\bf Sep. 2018 - Sep. 2021}\\
	En este proyecto se busca mejorar la eficiencia del simulador numérico SEMBA-UGRFDTD a la hora de tratar detalles finos del a estructura de un avión mediante el empleo de técnicas de submallado. El proyecto está financiado integramente por la empresa Airbus.
\end{minipage}	

\begin{minipage}{\textwidth}
	{\em COST ACCREDIT} \hfill {\bf Ene. 2017 - Dic. 2019}\\
	Este proyecto se enfocará los retos de la naturaleza estocástica y de banda ancha de las radiaciones electromgnéticas en los sistemas multi-funcionales presentes y futuros. Se incluye el modelado de las distribuciones estocásticas de campo y el desarrollo de métodos experimentales mediante sondas de campo cercano. El proyecto cuenta con más de 20 participantes de distintos países de la Unión Europea entre los que se incluye el GEG-UGR.
\end{minipage}

\begin{minipage}{\textwidth}
  {\em PRACE SREDIT} \hfill {\bf Ene. 2017 - Dic. 2019}\\
  El proyecto SREDIT (Simulations of Radiated Emissions in Densely Integrated Technologies) incluye la participación de la UGR y tres entidades europeas más. Este proyecto se enmarca dentro de la iniciativa PRACE (Partnership for Advanced Computing in Europe) que busca mejorar el impacto científico de la investigación ofreciendo recursos computacionales a los participantes.
\end{minipage}

\begin{minipage}{\textwidth}
  {\em UAVE-3} \hfill {\bf Ene. 2017 - Dic. 2019}\\
  Los socios de este proyecto (INTA, Universidad de Granada, UPC) y las empresas interesadas en el mismo (AIRBUS DS, ONERA y CIMNE) tienen como objetivo desarrollar metodologías innovadoras de análisis y pruebas mediante la integración de métodos avanzados de modelado y simulación electromagnética que puedan aplicarse a todo el ciclo de vida de la nueva generación de UAVs (\textit{Unmanned Air Vehicles}) incluyendo el diseño, la creación de prototipos, la certificación, el mantenimiento y la revisión, asegurando en todo momento la aeronavegabilidad.
\end{minipage}

\begin{minipage}{\textwidth}
  {\em UAVEMI} \hfill {\bf Ene. 2014 - Dic. 2016}\\
  En este proyecto se analiza la inmunidad electromagnética de vehículos aereos no tripulados (UAV) bajo los efectos de impacto de rayos y de campos radiados de alta intensidad (HIRF). Está en desarrollo mediante una colaboración entre la Universidad Politécnica de Cataluña (UPC), el Instituto Nacional de Técnica Aeroespacial (INTA) y la Universidad de Granada.
\end{minipage}

\begin{minipage}{\textwidth}
 {\em MORFEO} \hfill {\bf Sep. 2014 - Dic. 2015}\\
 El proyecto MORFEO busca evaluar la viabilidad de nuevas técnicas de enmallado para protección EMC de cables en aviones. Está co-financiado por el grupo Airbus y la Junta de Andalucía.
\end{minipage}

\begin{minipage}{\textwidth}
{\em MANIAS} \hfill {\bf Dic. 2014 - Dic. 2015}\\
El proyecto MANIAS consiste en desarrollar, fabricar y evaluar antenas multifuncionales que queden integradas en superficies aerodinámicas. Ha sido desarrollado mediante una colaboración con el grupo Airbus.
\end{minipage}

\begin{minipage}{\textwidth}
{\em A-UGRFDTD} \hfill {\bf Sep. 2012 - Sep. 2015}\\
Este proyecto consiste en el desarrollo de un simulador FDTD conformado y completamente funcional. 
Está financiado por el grupo {\it Airbus} y la Universidad de Granada.
\end{minipage}

{\em TERALAB} \hfill {\bf Nov. 2009 - Nov. 2013}\\
El proyecto TeraLab tiene como objetivo el desarrollo de un laboratorio numérico para el diseño y simulación de tecnología de terahercios. 
Para hacerlo, se desarrollaron modelos físicos en ese rango de frecuencias que fueron posteriormente integrados dentro del simulador.

{\em HIRF: SE} \hfill {\bf Dic. 2008 - Jun. 2013}\\
El proyecto HIRF-SE (High Intensity Radiated Fields: Synthetic Environment) busca desarrollar un framework computacional para simular fenómenos electromagnéticos durante la fase de desarrollo de aeronaves. Este proyecto estuvo financiado por la Comisión Europea y 44 participantes de la Unión Europea. La contribución de nuestro grupo consistió en un simulador FDTD volúmico completamente funcional que sería integrado dentro de este framework.

{\em TERASENSE} \hfill {\bf Oct. 2008 - Mar. 2012}\\
El objetivo de Terasense (Tecnología de Terahercios para aplicaciones de detección electromagnética) fue desarrollar dispostiivos que, funcionando en este rango de frecuencias, fuesen capaces de detectar objetos o substancias en tiempo real. Este proyecto fue financiado por el programa de investigación español CONSOLIDER.

{\em Efectos biológicos de la Radiofrecuencia} \hfill {\bf Jul. 2008 - Dic. 2008}\\
El objetivo de este proyecto fue evaluar el SAR (Specific Absorption Rate) de distintos tejidos biológicos.
Durante el transcurso de este proyecto, se utilizaron distintas herramientas de electromagnetismo computacional, particularmente HFSS y Semcad. El proyecto fue financiado por la Agencia Europea de Defensa (EDA).

\begin{minipage}{\textwidth}
{\em Técnicas GPR para valoración de patrimonio arquitectónico} \hfill {\bf Jul. 2008 - Mar. 2010}\\
En este proyecto se emplearon y desarrollaron técnicas de Radar de penetración de tierra (GPR) para evaluar el estado de conservación de parte del patrimonio arquitectónico andaluz. El proyecto fue financiado por la consejería de turismo y comercio. Fue llevado a cabo por el Centro Tecnológico del Mármol de Murcia y la Universidad de Granada.
\end{minipage}

{\em Técnicas de optimización para antenas UWB} \hfill {\bf Oct. 2007 - Dic. 2010}\\
En este proyecto se uso el software CST junto con MatLab y desarrollos de software propios para el diseño de antenas de banda ultra-ancha. Fue llevado a cabo por una colaboración de cuatro universidades españolas (UGR, UPC, UPV y UPCT) y estuvo financiado por el programa nacional de investigación y desarrollo.

\end{resume}

\end{document}




